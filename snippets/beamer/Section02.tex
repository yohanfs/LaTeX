\section{Introduction to Singular Value Decomposition}                                               

\begin{frame}[t]{Introduction to Singular Value Decomposition}

    \begin{itemize}
        \item Singular Value Decomposition (SVD) is a factorization of a matrix.
            It generalizes the eigenvalue based decomposition of a square matrix with an
            orthonormal eigenbasis to any $m~\times~n$
        \item SVD is a heart of huge companies like Google, Youtube, Amazon,
            Facebook, and many more.
        \item Applications:

            \begin{itemize}
                \item image compression
                \item empirical orthogonal function (EOF) analysis    
            \end{itemize}

    \end{itemize}

\end{frame}


\section{Eigenvalue Based Decomposition}

\begin{frame}[t]{Eigenvalue Based Decomposition}
%	\begin{figure}[h]
%		\centering
%		\includegraphics[scale=0.30]{Sct02_Projection_Plane.pdf}
%%		\caption{Rotational Kinetic Energy of a Rigid Body.}
%	\end{figure}
	\begin{itemize}
		\item Let us consider one example in previous chapter as follows:
		\begin{equation*}
			\begin{split}
				D =
				\begin{bmatrix}
					1 & 2 \\ 0 & -1
				\end{bmatrix}
				\quad \rightarrow \quad
				\underbrace{
				\begin{bmatrix}
					1 & 2 \\ 0 & -1
				\end{bmatrix}
				}_D
				=
				\underbrace{
				\begin{bmatrix}
					1 & 1 \\ 0 & -1
				\end{bmatrix}
				}_{X}
				\underbrace{
				\begin{bmatrix}
					1 & 0 \\ 0 & -1
				\end{bmatrix}
				}_{\Lambda}
				\underbrace{
				\begin{bmatrix}
					1 & 1 \\ 0 & -1
				\end{bmatrix}
				}_{X^{-1}}
			\end{split}
		\end{equation*}
		\item Based on this decomposition, matrix $D$ can be written as :
		\begin{equation*}
			\begin{bmatrix}
				1 & 2 \\ 0 & -1
			\end{bmatrix}
			=
			(1)
			\begin{bmatrix}
				1 \\ 0
			\end{bmatrix}
			\begin{bmatrix}
				1 & 1
			\end{bmatrix}
			+
			(-1)
			\begin{bmatrix}
				1 \\ -1
			\end{bmatrix}
			\begin{bmatrix}
				0 & -1
			\end{bmatrix}
		\end{equation*}
		\item Unfortunately, this decomposition (eigenvalue based decomposition) can only work for \textbf{invertible matrix}.
		\item \textbf{Question :} Can the other type of matrix also be decomposed into this kind of matrix multiplication?
		\item The answer is \textbf{Yes}, by using a technique called \textbf{Singular Value Decomposition (SVD)}.
	\end{itemize}
\end{frame}

\section{Singular Value Decomposition}

\begin{frame}[t]{Singular Value Decomposition}
%	\begin{figure}[h]
%		\centering
%		\includegraphics[scale=0.30]{Sct02_Projection_Plane.pdf}
%%		\caption{Rotational Kinetic Energy of a Rigid Body.}
%	\end{figure}
	\begin{itemize}
		\item In eigenvalue and eigenvector based decomposition, the basic equation is :
		\begin{equation*}
            A x_{\text{i}}  = x_{\text{i}} \lambda_{\text{i}}
			\quad \rightarrow \quad
			A X = X \Lambda
			\quad \rightarrow \quad
			A = X \Lambda X^{-1}
		\end{equation*}
		\item Similarly, in SVD, the basic equation that we use is :
		\begin{equation*}
            A v_{\text{i}} = u_{\text{i}} \sigma_{\text{i}}
			\quad \rightarrow \quad
			A V = U \Sigma
			\quad \rightarrow \quad
			A = U \Sigma V^{-1}
		\end{equation*}
		\item \textbf{Question : } How to determine the value of $V$, $U$ and $\Sigma$?
	\end{itemize}
\end{frame}

\begin{frame}[t]{Singular Value Decomposition}
%	\begin{figure}[h]
%		\centering
%		\includegraphics[scale=0.30]{Sct02_Projection_Plane.pdf}
%%		\caption{Rotational Kinetic Energy of a Rigid Body.}
%	\end{figure}
	\begin{itemize}
		\item Let us consider a special case where :
		\begin{enumerate}
			\item Matrix $U$ is chosen to be \textbf{orthonormal matrix}, so that :
			\begin{equation*}
                U^{\text{T}} = U^{-1}
				\quad \rightarrow \quad
                U^{\text{T}} U = I
			\end{equation*}
			\item AND Matrix $V$ is chosen to be \textbf{orthonormal matrix}, so that :
			\begin{equation*}
                V^{\text{T}} = V^{-1}
				\quad \rightarrow \quad
                V^{\text{T}} V = I
			\end{equation*}
		\end{enumerate}
		\item As the result :
		\begin{equation*}
			A = U  \Sigma V^{-1} \quad \rightarrow \quad A = U  \Sigma V^{T}
		\end{equation*}
	
    \item Consider to multiply $A^{\text{T}}$ and $A$

        \begin{equation*}
            A^{\text{T}} A = V \Sigma^2 V^{\text{T}} \quad \rightarrow \quad
            A^{\text{T}} A V = V \Sigma^2 
		\end{equation*}

    \item Similarly, multiply $A$ and $A^{\text{T}}$ 
		\begin{equation*}
            A A^{\text{T}} = U \Sigma^2 U^{\text{T}} \quad \rightarrow \quad A
            A^{\text{T}} U = U \Sigma^2 
		\end{equation*}

    \item Based on above results, multiplication between $A$ and $A^{\text{T}}$
        is similar to the problem of eigenvalues and eigenvectors. 

	\end{itemize}
\end{frame}

\begin{frame}[t]{Singular Value Decomposition}
%	\begin{figure}[h]
%		\centering
%		\includegraphics[scale=0.30]{Sct02_Projection_Plane.pdf}
%%		\caption{Rotational Kinetic Energy of a Rigid Body.}
%	\end{figure}
	\begin{itemize}
        \item $\Sigma^2$ is the \textbf{eigenvalues matrix} of $A A^{\text{T}}$ and
            $A^{\text{T}} A$ and  $\Sigma$ is the \textbf{eigenvalues matrix} of
            $A$.

        \item $\Sigma^2$ and $\Sigma$ are defined as  
        
        \begin{equation*}
            \underbrace{
			\begin{bmatrix}
				\lambda_1 & & & & & \\
				& \ddots & & & &  \\
				& & \lambda_r & & & \\
				& & & 0 & & \\
				& & & & \ddots & \\
				& & & & & 0\\
            \end{bmatrix}}_{\Sigma^2}
            \quad \quad
            \underbrace{
            \begin{bmatrix}
				\sigma_1 & & & & & \\
				& \ddots & & & &  \\
				& & \sigma_r & & & \\
				& & & 0 & & \\
				& & & & \ddots & \\
				& & & & & 0\\
            \end{bmatrix}}_{\Sigma}
        \end{equation*}
            where $\sigma$'s are \textbf{singular values} $\quad \rightarrow \quad$  $\sigma_{\text{i}}=\sqrt{\lambda_{\text{i}}}$
    \end{itemize}
\end{frame}

\begin{frame}[t]{Singular Value Decomposition}
    \begin{itemize}
        \item $AA^{\text{T}}$ and $A^{\text{T}}A$ are automatically symmetric.
            \begin{itemize}
                \item $u$'s are called \textbf{left singular vector} (unit eigenvectors
                    of $AA^{\text{T}}$)     
                \item $v$'s are called \textbf{right singular vector} (unit eigenvectors
                    of $A^{\text{T}}A$)     
                \item $||u_{\text{i}}||=1$ and $||v_{\text{i}}||=1$  
                \item $u$'s and $v$'s are orthogonal sets.  
            \end{itemize}

        \item $U$ and $V$ are in the following matrices  
        
        \begin{equation*}
			\underbrace{
			\begin{bmatrix}
				\vline &   & \vline &  & \vline \\
				u_1 & \cdots & u_r & \cdots & u_m \\
				\vline &   & \vline &  & \vline \\
			\end{bmatrix}
        }_{U}
            \quad \quad
			\underbrace{
			\begin{bmatrix}
                -v_1^{\text{T}}-  \\
				\vdots \\
                -v_r^{\text{T}}-\\
				\vdots\\
                -v_n^{\text{T}}-\\
			\end{bmatrix}
        }_{V^{\text{T}} = V^{-1}}
        \end{equation*}
    \end{itemize}
\end{frame}

\begin{frame}[t]{Singular Value Decomposition}
%	\begin{figure}[h]
%		\centering
%		\includegraphics[scale=0.30]{Sct02_Projection_Plane.pdf}
%%		\caption{Rotational Kinetic Energy of a Rigid Body.}
%	\end{figure}
	\begin{itemize}
		\item In general, using SVD, any matrix $A$ can be decomposed into the
            following matrices:
		\begin{equation*}
			A
			=
            \overbrace{
			\underbrace{
			\begin{bmatrix}
				\vline &   & \vline &  & \vline \\
				u_1 & \cdots & u_r & \cdots & u_m \\
				\vline &   & \vline &  & \vline \\
			\end{bmatrix}
        }_{U}}^{ m~x~m}
            \overbrace{
			\underbrace{
			\begin{bmatrix}
				\sigma_1 & & & & & \\
				& \ddots & & & &  \\
				& & \sigma_r & & & \\
				& & & 0 & & \\
				& & & & \ddots & \\
				& & & & & 0\\
			\end{bmatrix}
        }_{\Sigma}}^{m~x~n}
            \overbrace{
			\underbrace{
			\begin{bmatrix}
                -v_1^{\text{T}}-  \\
				\vdots \\
                -v_r^{\text{T}}-\\
				\vdots\\
                -v_n^{\text{T}}-\\
			\end{bmatrix}
        }_{V^{\text{T}} = V^{-1}}}^{n~x~n}
		\end{equation*}
		\item $A$ can be written as:
		\begin{equation*}
            A = U \Sigma V^{\text{T}} = u_1 \sigma_1 v_1^{\text{T}} + \cdots +
            u_{\text{r}} \sigma_{\text{r}} v_{\text{r}}^{\text{T}}
		\end{equation*}
    \item Generally, $\sigma_{\text{i}}$ is written in descending order so that
        $\sigma_1 \geq \sigma_2 \geq \cdots \geq \sigma_{\text{r}}$.
	\end{itemize}
\end{frame}

\section{Singular Value Decomposition - Example}

\begin{frame}[t]{Singular Value Decomposition - Example}
%	\begin{figure}[h]
%		\centering
%		\includegraphics[scale=0.30]{Sct02_Projection_Plane.pdf}
%%		\caption{Rotational Kinetic Energy of a Rigid Body.}
%	\end{figure}
	\begin{itemize}
		\item Find the matrices $U$, $\Sigma$, $V$ for
		\begin{equation*}
			A = 
			\begin{bmatrix}
				3 & 0 \\ 4 & 5
			\end{bmatrix}
		\end{equation*}

        \item Begin with $A^{\text{T}}A$ and $A^{\text{T}}A$:
    
        \begin{equation*}
            A^{\text{T}}A = 
			\begin{bmatrix}
				25 & 20 \\ 20 & 25
			\end{bmatrix}
            \quad \quad
            A A^{\text{T}} = 
			\begin{bmatrix}
				9 & 12 \\ 12 & 41
			\end{bmatrix}
		\end{equation*}

        \item Find $\Sigma^2$ of $A^{\text{T}}A$ and $\Sigma$ of $A$: 

        \begin{equation*}
			\Sigma^2 = 
			\begin{bmatrix}
				45 & 0 \\ 0 & 5
			\end{bmatrix}
        \quad            
			\Sigma = 
			\begin{bmatrix}
                \sqrt{45} & 0 \\ 0 & \sqrt{5}
			\end{bmatrix}
		\end{equation*}


\end{itemize}
\end{frame}

\begin{frame}[t]{Singular Value Decomposition - Example}
    \begin{itemize}
    \item Find $V$ of $A^{\text{T}}A$ with eigenvalues of $45$ and $5$: 

    \begin{equation*}
        \begin{bmatrix}
            25 & 20 \\
            20 & 25
        \end{bmatrix}
        \begin{bmatrix}
            1 \\ 1 
        \end{bmatrix}
        = 45
        \begin{bmatrix}
            1 \\ 1 
        \end{bmatrix}
        \quad \quad
        \begin{bmatrix}
            25 & 20 \\
            20 & 25
        \end{bmatrix}
        \begin{bmatrix}
            -1 \\ 1 
        \end{bmatrix}
        = 5
        \begin{bmatrix}
            -1 \\ 1 
        \end{bmatrix}
    \end{equation*}

    \begin{equation*}
        V=
        \begin{bmatrix}
            1 & -1 \\
            1 & 1 \\
        \end{bmatrix}
    \end{equation*}

    \item Normalize $V$ so that $||v_{\text{i}}||=1$
        \begin{itemize}
            \item Calculate current $||v_{\text{i}}||$
            \item $||v_{\text{1}}||^2=1^2+1^2$ $\quad \rightarrow \quad$
                $||v_{\text{1}}||=\sqrt 2$
            \item Hence, the scaling factor to normalize $V$ is $\frac{1}{\sqrt
                2}$
        \end{itemize}            

    \item Normalized $V$ is

        \begin{equation*}
        V=\frac{1}{\sqrt 2}
        \begin{bmatrix}
            1 & -1 \\
            1 & 1 \\
        \end{bmatrix}
    \end{equation*}

    \end{itemize}

\end{frame}    

\begin{frame}[t]{Singular Value Decomposition - Example}
\begin{itemize}
    \item Find $U$ of $AA^{\text{T}}$. It can use the previous procedures by
        solving eigenvectors of $AA^{\text{T}}$ or using the following
        procedures:
        \begin{equation*}
            U = A V \Sigma^{-1}
        \end{equation*}   
        \begin{equation*}
            U = \frac{1}{\sqrt 10}
            \begin{bmatrix}
                1 & -3 \\
                3 & 1 
            \end{bmatrix}
        \end{equation*}   
    \item The final results are

    \begin{equation*}
        \underbrace{
        \begin{bmatrix}
            3 & 0 \\
            4 & 5 \\
        \end{bmatrix}}_{A}
        =
        \underbrace{
        \frac{1}{\sqrt 10}
        \begin{bmatrix}
            1 & -3 \\
            3 & 1 
        \end{bmatrix}}_{U}
        \underbrace{
    	\begin{bmatrix}
            \sqrt{45} & 0 \\ 0 & \sqrt{5}
        \end{bmatrix}}_{\Sigma}
        \bigg(
        \underbrace{
        \frac{1}{\sqrt 2}
        \begin{bmatrix}
            1 & -1 \\
            1 & 1 \\
        \end{bmatrix}}_{V^{\text{T}}}\bigg)^{\text{T}}
    \end{equation*}

\end{itemize}
\end{frame}    
